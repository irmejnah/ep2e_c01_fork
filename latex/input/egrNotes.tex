
\section{How To Run Drones}

“Robots are a common sight and accepted fact of daily life within Eclipse Phase. Numerous varieties exist, from robopets to mechanical workers to warbots. If a job can be done more cheaply (and sometimes safely) by a bot, it usually is. The robots listed here are not generally used as synthetic shells by transhuman egos, often for cultural reasons (sleeving a case is bad enough, sleeving a creepy is just … weird), and they are not equipped to be sleeved into, though they can be remotely operated $\blacktriangleright$below. Any of these bots can be modified for use as a synthetic morph, however, by adding a cyberbrain system $\blacktriangleright$316. Most robots are intentionally built in non-humanoid forms so as not to confuse them with common synthmorphs and to defuse subconscious guilt transhumans might have at ordering anthropomorphic entities around. However, they all have some form of abstract “face” to interact with, so as not to be too machine-like.” \citep[pg. 346]{ep2e_1.1_2019}

According to the Discord Gear can be attached to drones, which matches the line where you can make a drone/ robot a Morph. So what should we (Infomorph) put onto our (main) drone in order to not run into all the “What can I do how from where?” problems?

As all bots already come with Mesh Inserts there should be no problem to take residence inside them. So this looks like a quite okay strategy -- especially combined with Ressources IV providing for upgrades to them as needed whenever in non-austere environments.

However they are by having a \gls{MeshInsert} space for one running Infomorph only. Therefore if running on there the device AI is inactive and therefore its skills cannot be used. \citep[pg. 244]{ep2e_1.1_2019} Upgrading with an \gls{Ecto} would get around this, but still mean a -10 malus on the skills (switching places in less than a Round should be possible to get around that though). There is no reason to use an \gls{Ecto} as \gls{Ghostrider Module} is the same rarity. Just upgrade the drone/ bot with one.

\subsection{Repairing Bots}

“Most synthmorphs and objects do not heal on their own and must be repaired. Some are equipped with self-repair systems that function the same as medichines ▶322. Fixer swarms ▶345 can repair damage and wounds the same as meds. Synthmorphs and objects can also be repaired in a nanofabrication machine with the appropriate blueprints; treat this the same as a healing vat. Synthmorphs can only benefit from one repair source at a time; use the fastest rate applicable.

Physical Repair: Manually repairing a synthmorph or object requires a Hardware Test using a field appropriate to the item (Hardware: Robotics for synthmorphs and bots, Hardware: Aerospace for aircraft, etc.) and a tool kit. Repair is a task action with a timeframe of 1 hour per 5 points of damage being restored, plus 8 hours per wound. Once completed, it is fully repaired. Apply appropriate modifiers based on conditions and available tools.” \citep[pg. 221]{ep2e_1.1_2019}

Do we have the Skills to heal and repair? If not can we/ should we fix that? The Explorenaut has some skill itself, so unless failing critically a lot (if that applies for repairs) it can fix itself in a few days time (mostly dependent on the number of wounds).


\subsection{Robot Ware}

“All robots are equipped with the same sensory systems you get with every morph: standard vision, hearing, touch, proprioception, balance, and so on — sometimes even smell. They are also all equipped with standard ware that facilitates their function, noted below. Individual bots have their own distinct ware payloads, as noted in their description. You may also modify bots with additional ware, just like a synthmorph. Some also use vehicle hardware $\blacktriangleright$350. Ware: 360-Degree Vision, Access Jacks, Bot AI, Lidar, Mesh Inserts, Puppet Sock” \citep[pg. 346]{ep2e_1.1_2019}

\subsection{Remote Operations}

“Any shell or biomorph with a puppet sock (included in most cyberbrains) can be remote controlled, either by a character or an AI. This requires a communications link between the teleoperator and the drone. The teleoperator controls the drone via entoptic interface and receives sensory input and other data via the drone’s mesh inserts. You can control a drone in autonomous mode or directly by jamming.

\paragraph{Autonomous Mode} Bot and vehicle shells are equipped with Bot/Vehicle ALIs $\blacktriangleright$326. In autonomous mode, the drone’s AI operates on its own, though it also follows commands issued verbally or via a communications link or entoptic control panel by an authorized entity. Issuing a one-sentence command is a mental quick action; commands can be issued to multiple drones at once. More complex commands may take longer, or can be prepared in advance with a Program Test. The AI may need to pass a COG Check to understand especially confusing or incomplete commands. Autonomous drones use their own Initiative, skills, and pools.

\paragraph{Jamming} You jam a drone by immersing yourself in an AR overlay of the drone’s body and sensorium, including proprioception — you essentially become the drone. Engaging or disengaging from jamming mode is a quick action. While immersed, you suffer a –30 modifier to Perceive Tests and physical actions involving your own morph. Jamming feels much like resleeving and you must make an Integration Test $\blacktriangleright$288 to acclimate to the drone’s form. Treat jamming the same as if you sleeved the drone; use the drone's pools instead of your own morph's. However, since jamming is slightly inferior to actual sleeving, you suffer a –10 modifier to all actions (unless you are equipped with drone rig ware $\blacktriangleright$320).” \citep[pg. 346]{ep2e_1.1_2019}

\subsection{Uparmoring The Bot}

A possible role, especially if connection loss ans sitting it out become a regular thing, for \texttt{\egr{}}\index{\egr{}} might be to uparmor the preferred drone and to be a heavy scout/ area tank kind of thing advancing ahead of the group, gathering and sharing information, and ask the enemy if they want to shoot the heaily armored drone or a real enemy (unless combined with “Gearing The Bot As Morph”).

While it is unclear when the Bot would go to half movement (to be on the safe side let's assume all them time if adding any armor for now) they should be able to add armor up to their durability. So for Explorenaut that allows both Light Combat Armor and Heavy Combat Armor \citep[p. 215]{ep2e_1.1_2019}. As they are not “Layering Armor” that should not even get a malus on checks, just being spending GP and be good. Their Discord even thinks that the movement reduction only applies to layering. ALso, the ALI aptitudes are 10. \myurl{https://discord.com/channels/575153209065340929/1082012122050990192/1175536739989856376}

\subsection{Gearing The Bot As Morph}

If \gls{Mesh} connectivity is suspected to become a problem \texttt{\egr{}}\index{\egr{}} might be best of with hosting itself on its preferred bot.


\section{Combat}

\texttt{Interface} works for active participation in combat sometimes. \qEnglish{Interface is also used to attack with non-portable weapon emplacements and the weapon systems of piloted (but not jammed or sleeved) vehicles and spacecraft (Weapon Systems ▶209).} \citep[p. 50]{ep2e_1.1_2019}


\section{Hacking and Mesh Actions}

“In the digital realm, everything has a vulnerability. Software is the classic plan that never survives contact with the enemy. Hackers are continuously scouting for flaws in code that will allow them to exploit it for unintended purposes. As quickly as these flaws are discovered and patched, new ones are discovered and leveraged for advantage.

Hackers routinely share, trade, and sell their exploits online. The best of these make their way into pre-packaged exploit app libraries — software tools that scan a target, openly or with subtlety, probe it for vulnerabilities, and automatically execute attacks. 

There are many methods you may use to hack a system. The first is to circumvent the authentication ▶246 on a legitimate account, but this requires background knowledge of the account and takes time and special effort. The second is to sniff the traffic of a legitimate user ▶245 and remotely spoof commands ▶247, which can be powerful but limited. The most common method of hacking is to directly gain access to the target (Intrusion ▶below) and subvert it from within (Subversion ▶262).” \citep[pg. 258]{ep2e_1.1_2019}

There ist kinds of Hacking, and there is defense against Hacking. Offensively there is Hacking a system \citep[pg. 258]{ep2e_1.1_2019}, Mindware Hacking \citep[pg. 266]{ep2e_1.1_2019}, and Surveillance Hacking \citep[pg. 270]{ep2e_1.1_2019}, and Countermeasures is defense against Hacking \citep[pg. 260]{ep2e_1.1_2019}. Some of it is resolved by Mesh Combat \citep[pg. 264]{ep2e_1.1_2019}.


\section{Clarifications}

\begin{enumerate}
    \item \texttt{Perceive}  for Perceive it is stated "use of your physical senses" (p. 51). If so how does \egr{} check for perception?

    \item \texttt{Resources} states that at level 4, which \egr{} has, it might have its own habitat or shuttle \citep[p. 75]{ep2e_1.1_2019}. Clarify, maybe need to pick a location.

    \item Is a Portable Server a thing available?
\end{enumerate}

\section{Leveling or Adjustments}

When acquiring Gear (pg. 312) one can acquire the \gls{Blueprint} as Multi-use increasing the Complexity by one category. “Apps of course, can go with you, and some subscriptions may last through the next mission.” \citep[p. 313]{ep2e_1.1_2019}

\begin{itemize}
    \item \texttt{Medicine} is maybe something we should have in the crew? Healing seems to work well enough, but some investigative stuff might be needed. \textbf{Medical: Biotech} “is to modify morphs or install implants.” \citep[p. 51]{ep2e_1.1_2019} So we need that to install the Ware we are printing. Though for the other usual application we'll need \textbf{Medical: Paramedic} or \textbf{Medical: Forensics}. There is no Medicine as a Know Skill, but \textbf{Biology}. So I would think becoming a really good surgeon and medic for the crew would for \egr{} be getting 20+ \textbf{Medical: Biotech} as that would be perfect with Taking Time, high \textbf{Medical: Paramedic} as that will be most likely be time critical not allowing to push it to max by Taking Time, and \textbf{Medical: Forensics} as a tertiary goal.
    \item \texttt{\gls{Ghostrider Module}} needed at least in austere environments. If we don't have a \gls{Blueprint} from starting equipment we need to accquire one ASAP.
\end{itemize}

\subsection{Plan B}

Get GP by dropping Tracker (2GP \citep[p. 326]{ep2e_1.1_2019}) and Multitasking (2GP \citep[p. 320]{ep2e_1.1_2019}) getting Medical/Repair (1GP \citep[p. 217]{ep2e_1.1_2019}), Self-Healing (1GP \citep[p. 217]{ep2e_1.1_2019}) and Light Combat Armor (2GP \citep[p. 215]{ep2e_1.1_2019}) or Medical/Repair (1GP \citep[p. 217]{ep2e_1.1_2019}) and \gls{Heavy Combat Armor} (3GP \citep[p. 215]{ep2e_1.1_2019}).

How does Armor take damage that would require Self-Healing (1GP \citep[p. 217]{ep2e_1.1_2019})?


\section{Forking/Copying}

Forking and Copying are used interchangingly it seems. The technical term should be Forking as in the chapter title “Forking \& Merging”, where it starts out as a quite normal everyday activity seemingly: “One of the repercussions of translating your mind into a digital format is that you can copy it just like any other file. Taking a copy of a transhuman ego and re-instancing it is called forking. To many, forking is an advantage, allowing you to multi-task and get more done. Need help on a project? Fork a few copies of yourself. Need to be in simultaneous meetings on Mars and Venus? Spin off a fork and egocast it to one while you handle the other — or send forks to both and go grab a drink.” \citep[pg. 292]{ep2e_1.1_2019}

The limitations are introduced as a practical matter first in the very next paragraph: “Forking, however, creates some thorny social and legal issues. Is your fork considered a separate person with full civil rights? Or are they your property? Is deleting them murder? Is your spouse married to your fork, or just you? If you and your fork have an argument and go your separate ways, who gets your stuff? Complicating this matter is the fact that the longer you and your fork are apart, the more your individual minds start to diverge, effectively becoming separate people. While forks can be re-merged into the originating ego, this becomes more difficult the longer they are apart.” \citep[pg. 292]{ep2e_1.1_2019}

It is then made clear that Beta Forks are commonly kept prepared on shelf: “Beta forks are partial copies. They are intentionally hobbled so as to not to be considered an equal to the originator, or to not have all your memories, for legal, social, and security reasons. Beta forks have most of the same skills as the original ego, though sometimes reduced. Their memories are also drastically curtailed, usually tailored to whatever task they are intended to perform. Many people keep pre-made beta forks on hand to activate as needed, deleting them afterwards.” \citep[pg. 292]{ep2e_1.1_2019}

So they are widely used, but why? What is this referring to with “legal, social, and security reasons”? While it is not immideately clear it not being further explained, and following the previously quoted paragraph, I think it is reasonable that this should be read in the same practicalities manner. If you don't fork safety first and use an Alpha Fork instead, what if you have differences with it and the legal status becomes relevant like in the ways mentioned above? This would however hardly be relevant within seconds or minutes, probably very unlikely within hours, especially if not sleeping in between (and Infomorph don't sleep). There needs to be some individual experience that shutters the original common understanding that led to doing the Forking in the first place in order to get differences.

The subsection “Forks And Society” should allow to test that assumption's validity. And they deliver some clarity: “The way forks are treated varies by polity and culture. In the Consortium, forks are considered property and must identify themselves as such; alpha forks are legally limited to a 48-hour existence. Morningstar is similar, but allows alphas to be granted limited personhood status after a one-week existence, permission from the originator, and an application for citizenship. Most autonomists treat alphas as full, separate individuals, but opinions are split on beta forks. The Jovians have outlawed forking entirely. Some polities require forks to be equipped with the auto-erase app ▶326 so that they automatically erase when they reach the legal divergence period.”

So we learn that in Jovian controlled areas Forking is just illegal in total. Little need for further interpretation. Fitting their philosophy this seems natural and should indeed also cover Beta Forking.

For the rest is pretty clearly telling me that Alpha Forking is legal in both Consortium and Morningstar controlled areas, with Consortium having the legal demand to delete within 48 hours and the other seemingly no limitations. This leaves open the question what the other Factions would have as legal frameworks, but given that they are more free and more technology accepting I'd argue for the Morningstar position, but certainly not less than Consortium law.

There is also the subsection “Handling Forks” on the very next page advicing: “GMs are encouraged to allow players to roleplay their character’s own forks. It is important to note, however, that even with alpha forks, once the fork and originating ego diverge, they develop onward as separate people. The events that shape the primary ego’s personality, character, and knowledge will not happen — or even if they do, probably not in the same way — to the fork, and vice versa. The exact dividing line between an ego and a fork is a central philosophical and legal debate among many transhumans. This means that the GM should not be afraid to pull a fork out of a player character’s hands and make them into an NPC if they start to diverge too greatly. Similarly, if a fork begins to learn information that the main character does not (yet) have access to, it is probably also better to run the fork as an NPC in order to avoid metagaming. It is entirely possible that a fork might decide that it will no longer obey the originating ego and carry about doing its own thing. This usually only occurs with alpha forks, who are essentially a full copy anyway, and as time passes the idea of merging back with the original ego becomes unappealing. Beta forks are quite aware of their nature as “incomplete” copies and are less likely to diverge and make a break for life on their own.”

That reads to me that Forking is a very normal standard table prodecure and just offers some methods of limits excessive use -- as indeed the rules should probably be a bit more restrictive to make it less of a natural thing to run a whole swarm all the time, which would not be good for the table.

The introduction into “Transhuman Tech” however sounds less permissive: “Forking is legally restricted in many polities, but it is a common practice to upload beta forks, egocast them to distant locales for important errands, and merge them with their alpha when they return.” \citep[pg. 286]{ep2e_1.1_2019}

Many is quite unclear. We have established that there is a major Faction, the Jovians, where all Forking is illegal, and that it is not at all in another, the Morningstar Constellation. I also would expect Forking a being that has a biological origin, and might be a Biomorph currently, would be seen somewhat different from an Infomorph, especially if that is orinally from AGI, and never has been uploaded -- as we currently assume for both our PCs, though according to the book that would be the exception.

Therefore I argue that Alpha Forking outside Jovian territories should be quite acceptable as standard procedures. Though nobody wants to build a wall -- of copies.\footnote{This makes more sense in German I suppose.} I actually think this is a bit to liberal in allowing the use for players. But while I question sending out the copies to do jobs in the wild, I do maintain that server use and such are obviously intented use. There is also a line in advancements where they just argue to not give extra points for the extra bandwidth achieved with Forking, which once more makes clear that Forking is intented as a standard method of play.


\section{PSI}

\paragraph{PSI Range} states “Psi-chi sleights only affect the async. Psi-gamma sleights may be used on the self or other biological life at a short distance.” \citep[pg. 276]{ep2e_1.1_2019}

So Infomorphs are incapable of using any PSI, but also immune to being targeted with any.
