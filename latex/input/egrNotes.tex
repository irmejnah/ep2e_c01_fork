
\subsection{How To Run Drones}

“Robots are a common sight and accepted fact of daily life within Eclipse Phase. Numerous varieties exist, from robopets to mechanical workers to warbots. If a job can be done more cheaply (and sometimes safely) by a bot, it usually is. The robots listed here are not generally used as synthetic shells by transhuman egos, often for cultural reasons (sleeving a case is bad enough, sleeving a creepy is just … weird), and they are not equipped to be sleeved into, though they can be remotely operated $\blacktriangleright$below. Any of these bots can be modified for use as a synthetic morph, however, by adding a cyberbrain system $\blacktriangleright$316. Most robots are intentionally built in non-humanoid forms so as not to confuse them with common synthmorphs and to defuse subconscious guilt transhumans might have at ordering anthropomorphic entities around. However, they all have some form of abstract “face” to interact with, so as not to be too machine-like.” \citep[pg. 346]{ep2e_1.1_2019}

According to the Discord Gear can be attached to drones, which matches the line where you can make a drone/ robot a Morph. So what should we (Infomorph) put onto our (main) drone in order to not run into all the “What can I do how from where?” problems?

As all bots already come with Mesh Inserts there should be no problem to take residence inside them. So this looks like a quite okay strategy -- especially combined with Ressources IV providing for upgrades to them as needed whenever in non-austere environments.

\subsubsection{Repairing Bots}

“Most synthmorphs and objects do not heal on their own and must be repaired. Some are equipped with self-repair systems that function the same as medichines ▶322. Fixer swarms ▶345 can repair damage and wounds the same as meds. Synthmorphs and objects can also be repaired in a nanofabrication machine with the appropriate blueprints; treat this the same as a healing vat. Synthmorphs can only benefit from one repair source at a time; use the fastest rate applicable.

Physical Repair: Manually repairing a synthmorph or object requires a Hardware Test using a field appropriate to the item (Hardware: Robotics for synthmorphs and bots, Hardware: Aerospace for aircraft, etc.) and a tool kit. Repair is a task action with a timeframe of 1 hour per 5 points of damage being restored, plus 8 hours per wound. Once completed, it is fully repaired. Apply appropriate modifiers based on conditions and available tools.” \citep[pg. 221]{ep2e_1.1_2019}

Do we have the Skills to heal and repair? If not can we/ should we fix that? The Explorenaut has some skill itself, so unless failing critically a lot (if that applies for repairs) it can fix itself in a few days time (mostly dependent on the number of wounds).


\subsubsection{Robot Ware}

“All robots are equipped with the same sensory systems you get with every morph: standard vision, hearing, touch, proprioception, balance, and so on — sometimes even smell. They are also all equipped with standard ware that facilitates their function, noted below. Individual bots have their own distinct ware payloads, as noted in their description. You may also modify bots with additional ware, just like a synthmorph. Some also use vehicle hardware $\blacktriangleright$350. Ware: 360-Degree Vision, Access Jacks, Bot AI, Lidar, Mesh Inserts, Puppet Sock” \citep[pg. 346]{ep2e_1.1_2019}

\subsubsection{Remote Operations}

“Any shell or biomorph with a puppet sock (included in most cyberbrains) can be remote controlled, either by a character or an AI. This requires a communications link between the teleoperator and the drone. The teleoperator controls the drone via entoptic interface and receives sensory input and other data via the drone’s mesh inserts. You can control a drone in autonomous mode or directly by jamming.

\paragraph{Autonomous Mode} Bot and vehicle shells are equipped with Bot/Vehicle ALIs $\blacktriangleright$326. In autonomous mode, the drone’s AI operates on its own, though it also follows commands issued verbally or via a communications link or entoptic control panel by an authorized entity. Issuing a one-sentence command is a mental quick action; commands can be issued to multiple drones at once. More complex commands may take longer, or can be prepared in advance with a Program Test. The AI may need to pass a COG Check to understand especially confusing or incomplete commands. Autonomous drones use their own Initiative, skills, and pools.

\paragraph{Jamming} You jam a drone by immersing yourself in an AR overlay of the drone’s body and sensorium, including proprioception — you essentially become the drone. Engaging or disengaging from jamming mode is a quick action. While immersed, you suffer a –30 modifier to Perceive Tests and physical actions involving your own morph. Jamming feels much like resleeving and you must make an Integration Test $\blacktriangleright$288 to acclimate to the drone’s form. Treat jamming the same as if you sleeved the drone; use the drone's pools instead of your own morph's. However, since jamming is slightly inferior to actual sleeving, you suffer a –10 modifier to all actions (unless you are equipped with drone rig ware $\blacktriangleright$320).” \citep[pg. 346]{ep2e_1.1_2019}

\subsubsection{Uparmoring The Bot}

A possible role, especially if connection loss ans sitting it out become a regular thing, for \texttt{\egr{}}\index{\egr{}} might be to uparmor the preferred drone and to be a heavy scout/ area tank kind of thing advancing ahead of the group, gathering and sharing information, and ask the enemy if they want to shoot the heaily armored drone or a real enemy (unless combined with “Gearing The Bot As Morph”).

While it is unclear when the Bot would go to half movement (to be on the safe side let's assume all them time if adding any armor for now) they should be able to add armor up to their durability. So for Explorenaut that allows both Light Combat Armor and Heavy Combat Armor \citep[p. 215]{ep2e_1.1_2019}. As they are not “Layering Armor” that should not even get a malus on checks, just ebing spending GP and be good. Their Discord even thinks theat the movement reduction only applies to layering. ALso, the ALI aptitudes are 10. \myurl{https://discord.com/channels/575153209065340929/1082012122050990192/1175536739989856376}

\subsubsection{Gearing The Bot As Morph}

If \gls{Mesh} connectivity is suspected to become a problem \texttt{\egr{}}\index{\egr{}} might be best of with hosting itself on its preferred bot.


\subsection{Combat}

\texttt{Interface} works for active participation in combat sometimes. \qEnglish{Interface is also used to attack with non-portable weapon emplacements and the weapon systems of piloted (but not jammed or sleeved) vehicles and spacecraft (Weapon Systems ▶209).} \citep[p. 50]{ep2e_1.1_2019}


\subsection{Clarifications}

\texttt{Perceive}  for Perceive it is stated "use of your physical senses" (p. 51). If so how does \egr{} check for perception?

\texttt{Resources} states that at level 4, which \egr{} has, it might have its own habitat or shuttle \citep[p. 75]{ep2e_1.1_2019}. Clarify, maybe need to pick a location.


\subsection{Leveling or Adjustments}

\texttt{Medicine} is maybe something we should have in the crew? Healing seems to work well enough, but some investigative stuff might be needed.

\subsubsection{Plan B}

Get GP by dropping Tracker (2GP \citep[p. 326]{ep2e_1.1_2019}) and Multitasking (2GP \citep[p. 320]{ep2e_1.1_2019}) getting Medical/Repair (1GP \citep[p. 217]{ep2e_1.1_2019}), Self-Healing (1GP \citep[p. 217]{ep2e_1.1_2019}) and Light Combat Armor (2GP \citep[p. 215]{ep2e_1.1_2019}) or Medical/Repair (1GP \citep[p. 217]{ep2e_1.1_2019}) and Heavy Combat Armor (3GP \citep[p. 215]{ep2e_1.1_2019}).

How does Armor take damage that would require Self-Healing (1GP \citep[p. 217]{ep2e_1.1_2019})?