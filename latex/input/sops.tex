
\section{Post Egocasting}

\begin{enumerate}
    \item If \gls{Private Server} available create \gls{Fork} as \gls{Active Defender} for the \gls{Private Server} as the default \gls{Home Device}.
\end{enumerate}


\section{Nanofabrication}

\begin{enumerate}
    \item Anything being produced needs a \gls{Nanofabricator}, Raw Materials \citep[p. 314]{ep2e_1.1_2019}, and a \gls{Blueprint}.

    If all three have been acquired: “Printing an item with a blueprint does not require a test in most circumstances — the blueprint is enough, the nanofabber does the rest. The exceptions are print jobs that are exotic, complicated, or have limited feedstock or incomplete or suspect blueprints. In those cases, a Program Test is in order. Note that most nanofabbers have a built-in ALI with a Program 30 (Nanofabrication 40) skill. Once the raw materials and blueprints are in, nanofabrication is simply a matter of time. The printing time is based on the item’s complexity, the same as acquiring gear (2 hours for Minor, 8 Moderate, 24 Major). GMs may feel free to modify this period as appropriate for the object.” \citep[pg. 314]{ep2e_1.1_2019}

    \item “Raw materials are generally easy to acquire, as most nanofabricators are equipped with disassembler units that will break down just about anything into its constituent molecules. Feedstock is also easily/cheaply available in most habitats, either in bulk blocks or via utility feedline pipes direct to residential units. Many habitats route their recycling and waste products directly into disassemblers. At the GM’s discretion, certain nanofab designs may require hardto- get rare materials. This is especially true of explosives, ammunition, sensors, and some scientific gear and electronics. This could include rare heavy metals (platinum, tungsten, depleted uranium), uncommon radioisotopes, fissionables, or antimatter. Any print job that requires massive amounts of a particular material (notably water) can also be difficult, given the sometimes limited supplies. Acquiring rare materials might require a major favor or Resources expenditure — or could be an adventure unto itself.” \citep[pg. 314]{ep2e_1.1_2019}
    \item Besides acquiring \glsdisp{Blueprint}{Blueprints} through Acquiring Gear resources “[a]ll starting gear includes the physical gear item and the digital blueprint for nanofabrication.” \citep[pg. 68]{ep2e_1.1_2019}.

    \glsdisp{Blueprint}{Blueprints} can also be developed by the player characters: “If you don’t have a blueprint, you can make one. This requires two skill tests. The first is a skill test appropriate to the item type: Hardware: Electronics for personal electronics, Medicine: Pharmacology for a drug, Medicine: Biotech for bioware, etc. If successful, this provides the knowledge needed for the design. Note that programmers can collaborate with others who have the necessary skills, including their muse or an ALI. Following this, you need a Program Test to actually code it. Both of these tests together constitute a single task action. The timeframe is the same as acquiring gear ▶312, but in months instead of hours (2 months for Minor, 8 months for Moderate, 24 months for Major). Most programmers use forks and time-accelerated simulspaces to speed the process, however, so a subjective week of programming can be done in about an hour of real time. Only superior results from the Program Test reduce the timeframe.” \citep[pg. 314]{ep2e_1.1_2019}.

    It is also possible to remove the restrictions from DRM protected one- or limited-use \glsdisp{Blueprint}{Blueprints}: “The digital restrictions that prevent single- and multi-use blueprints from being shared can be defeated, given enough time. This requires a Program Test task action with a timeframe of 6 months (or 3 days in a time-accelerated VR). If you succeed, the blueprint can be re-used and copied freely.” \citep[pg. 314]{ep2e_1.1_2019}.
\end{enumerate}