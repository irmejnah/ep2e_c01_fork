\newglossaryentry{Active Defender}{
    name={Active Defender},
    description={see \gls{Active Defense}},
    sort={Active Defender}
}

\newglossaryentry{Active Defense}{
    name={Active Defense},
    description={“As a complex action, a system defender can assume control of a system’s firewall defenses for one action turn. While engaged in active defense, use the defender’s Infosec skill in place of Firewall rating for Hacking Tests. The defender may use pools on these tests. Only one defender can engage in active defense at a time.” \citep[pg. 260]{ep2e_1.1_2019}},
    sort={Active Defense}
}

\newglossaryentry{Acumen}{
    name={Acumen},
    description={
        “You have a keen intellect. Add +5 per level to your COG Checks.” \citep[pg. 72]{ep2e_1.1_2019}},
    sort={Acumen}
}

\newglossaryentry{Ada}{
    name={Ada},
    description={...},
    sort={Ada}
}

\newglossaryentry{AGI}{
    name={AGI},
    plural={AGIs},
    description={
        Artificial General Intelligence. An artificial intelligence that is in the human sense generally intelligent, meaning it can handle tasks it was not trained for in a common sense intelligent manner.
    },
    sort={AGI}
}

\newglossaryentry{Anonymizer}{
    name={Anonymizer},
    plural={Anonymizers},
    description={
        “You have an account with an anonymizing proxy service that masks your Mesh ID ▶246.” \citep[pg. 315]{ep2e_1.1_2019}
    },
    sort={Anonymizer}
}

\newglossaryentry{Blueprint}{
    name={Blueprint},
    plural={Blueprints},
    description={
        “In order to print something on a nanofabricator, you need a blueprint. Most have a small inventory of essential items. Others may have blueprints in storage, but locked to specific users (these can be hacked per standard Infosec intrusion tests).\gPar{}
        Nanofab blueprints come in three forms: single use, multi use, and open source,
        \begin{itemize}
            \item \textbf{Single-Use Blueprints:} Most single-use blueprints are accessed via a digital distribution platforms which use digital rights management (DRM) to protect copyright. This means you can access the file to print it once, but it cannot be copied or printed again. If you need a file you can print without mesh access, some single-use blueprints are available as downloadable digital files that block efforts to copy them and erase themselves after use. The inner-system hypercorps and capitalist economies use these methods to control scarcity and keep you dependent. Single-use blueprints are acquired with GP, rep, or Resources trait, using the item’s complexity.
            \item \textbf{Limited-Use Blueprints:} A few hypercorps and transitional economies allow limited licenses. These work the same as single-use blueprints, except that if you recycle the item by disassembling it in a licensed fabber, you receive a credit which you can then use to print the item again. This is ideal for those who egocast often but wish to bring their gear with them.
            \item \textbf{Multi-Use Blueprints:} A multi-use blueprint is more akin to purchasing a personal license. You can print the item repeatedly and copy the blueprint, but it is keyed to your ID. These are also acquired via GP, rep, or Resources trait. For a non-erasing, re-usable, multi-use blueprint, increase the complexity by one step (Minor becomes Moderate, Moderate becomes Major). Multi-use blueprints for Major complexity items are rare and largely unavailable.
            \item \textbf{Open-Source Blueprints:} In autonomist regions and many other areas, some blueprints are freely available from open-source libraries. These blueprints are re-usable, can be copied, and do not require your authenticated ID to use. A Research Test (using the Rep Modifiers on the Rep Tests table ▶308) or (@-rep, i-rep, r-rep, or x-rep) favor equal to the item’s complexity plus one step will get you an open-source blueprint, even for things like weapons and drugs. The drawback is that open-source prints are sometimes less reliable, more experimental, or carry hidden malware payloads. Every open-source blueprint acquired via Research (not rep) has a 20\% chance of being unreliable: a –10 to –30 modifier, unexpected bugs, dangerous malware, etc. These blueprints are also illegal in certain inner-system polities (many are cracked proprietary designs) and may not work on inner-system fabbers.
        \end{itemize}” \citep[pg. 314]{ep2e_1.1_2019}
    }
}

\newglossaryentry{Burner Mesh ID}{
    name={Burner Mesh ID},
    plural={Burner Mesh IDs},
    description={
        “The easiest method of anonymizing your mesh activity is to deploy a “burner” mesh ID for each separate online transaction. Burner IDs are meant to be used one time and then erased. Though illegal in many jurisdictions, they are popular with criminals and anyone wanting a low profile. You may use burner IDs simultaneously as your real mesh ID or other burners for different online connections (though this is considered poor opsec, as they may be correlated). Burner IDs are ideal in that they are used only for short periods. Though they may be traced or even sniffed like other mesh IDs, the trail will end when they stop being used. Though they are easy to deploy, they are not fool-proof, and some care must be taken to avoid leaking your real mesh ID or carelessly associating it with accounts or other data that may betray you. Keep in mind that a burner ID will not help you if you are traced or sniffed while you are still using it. A resourceful opponent may also be able to correlate burner ID use with physical surveillance footage or other mesh ID logs. If someone attempts to trace you using a burner mesh ID you have discarded, they must make a Research Test at –30 against your Infosec skill.” \citep[pg. 257]{ep2e_1.1_2019}
    },
    sort={Burner Mesh ID}
}

\newglossaryentry{Common Sense}{
    name={Common Sense},
    plural={Common Sense},
    description={
        “Your innate sense of judgment cuts through distractions and other factors that might cloud a decision. Once per game session, you may ask the GM what choice to make or course of action to take; the GM will give you solid advice based on what your character knows. Alternatively, if you are about to make a disastrous decision, the GM can use your free hint to warn you that you are making a grave mistake.” \citep[pg. 73]{ep2e_1.1_2019}
    },
    sort={Complementary Skill}
}

\newglossaryentry{Complementary Skill}{
    name={Complementary Skill},
    plural={Complementary Skills},
    description={
        “In certain cases, Know skills can aid Active skill tests with a complementary skill bonus modifier. This should only apply to situations where the Know skill provides information that would not normally be encompassed by the Active skill. For example, Know: Religious Cults could be applied as a complementary skill when trying to Persuade a religious brinker group, but Know: Engineering is not complementary to a Hardware: Industrial Test to repair part of a habitat, because the Hardware skill already incorporates such engineering knowledge. The bonus for a complementary skill is based upon its rating, as noted on the Complementary Skill Bonus table.” (40-59: +10, 60-79: +20, 80+: +30) \citep[pg. 53]{ep2e_1.1_2019}
    },
    sort={Complementary Skill}
}

\newglossaryentry{Composure}{
    name={Composure},
    plural={Composure},
    description={
        “\textbf{Ego Trait CP Cost: 2}
        \gPar{}
        Your mental equilibrium is well tuned. You receive a +5 bonus to your Lucidity. This also affects derived stats based on Lucidity; increase your Trauma Threshold by 1 and your Insanity Rating by 10.” \citep[pg. 73]{ep2e_1.1_2019}
    },
    sort={Composure}
}

\newglossaryentry{Cortical Stack}{
    name={Cortical Stack},
    plural={Cortical Stacks},
    description={
        “This diamond memory module is implanted at the base of the skull, where the brain stem and spinal cord connect (or in similar locations on synthmorphs). It is linked to a neural lace network of nanobots saturating the brain that monitor synaptic connections, brain architecture, and chemical levels. It effectively takes a snapshot of your brain every second, making a digital backup of your ego, right up until the moment you die. It also backs up your muse. If you are killed, this stack can be used to retrieve a backup of your ego ▶287 so that you can be re-instantiated. Cortical stacks do not have external or wireless access (for security purposes), they must be surgically removed.” \citep[pg. 316]{ep2e_1.1_2019}
    },
    sort={Cortical Stack}
}

\newglossaryentry{Crypto}{
    name={Crypto},
    plural={Cryptos},
    description={
        “This app generates key pairs, encrypts messages using
    public keys, and decrypts with secret keys (Encryption ▶247).” \citep[pg. 326]{ep2e_1.1_2019}
    },
    sort={Crypto}
}

\newglossaryentry{Cyberbrain}{
    name={Cyberbrain},
    plural={Cyberbrains},
    description={
        “Cyberbrains are specialized hardware for running virtual mind-states, allowing an ego or ALI to control a pod or synthmorph. Modeled on biological brains, cyberbrains have a holistic architecture and serve as the command node and central processing point for sensory input and decision-making. As hosts, only one infomorph can inhabit a cyberbrain at a time. Any ego within a cyberbrain can move or copy itself as an infomorph to another device. By default, your ego will manifest in a digimorph ▶67, unless you have another type of infomorph available. Cyberbrains are vulnerable to hacking (Mindware Hacking ▶266). All cyberbrains are equipped with access jacks, mnemonics, and a puppet sock for remote control, at no additional cost.” \citep[pg. 316]{ep2e_1.1_2019}
    },
    sort={Cyberbrain}
}

\newglossaryentry{Defaulting}{
    name={Defaulting},
    description={
        “If you lack the skill you need to make a test, you can rely on your character’s innate talents and default to the skill’s linked aptitude instead (Aptitudes ▶36). For example, if you lack Guns skill, you can still shoot using your Reflexes aptitude as the target number. There is no modifier for defaulting on a skill test, but critical successes are ignored. In some cases, the GM may allow you to default to a related skill. For example, someone trying to repair a gun without any Hardware skills could default to their Guns skill. In this case, a –10 to –30 modifier applies, depending on how closely the skills are related. The GM may decide that some tests require specialized knowledge or training and so cannot be defaulted on.” \citep[pg. 31]{ep2e_1.1_2019}
    },
    sort={Defaulting}
}

\newglossaryentry{DocBot}{
    name={DocBot},
    plural={DocBots},
    description={
        “These wheeled medical robots are designed to tend to and transport injured or sick people. They carry a fabber for medical supplies and pharmaceuticals, miscellaneous medical gear, a secure container for carrying heads, and 4–8 articulated arms for conducting remote surgery. They are often loaded up with healing spray and meds (acquired separately).” \citep[pg. 347]{ep2e_1.1_2019}
    },
    sort={DocBot}
}

\newglossaryentry{Drone Rig}{
    name={Drone Rig},
    plural={Drone Rigs},
    description={
        “This simsense augmentation gives you better control when jamming drones (Remote Operations ▶346). You ignore the –10 modifier for jamming.” \citep[pg. 320]{ep2e_1.1_2019}
    },
    sort={Dron Rig}
}

\newglossaryentry{Durability}{
    name={Durability},
    plural={Durability},
    description={
        “Your physical health is measured by your morph’s Durability stat. For biomorphs, this represents the point at which accumulated injuries overwhelm and incapacitate you. Once your total damage points equal or exceed your Durability, you immediately collapse from exhaustion and physical abuse. You gain the unconscious condition ▶226 and cannot be revived until your damage points are reduced below your Durability, either from medical care or natural healing. For synthmorphs, Durability represents structural integrity. You become physically disabled when accumulated damage points reach your Durability and your software mind-state crashes. Even if you are unconscious/disabled, your mesh inserts may still function (Damage and Infomorph Riders ▶265).” \citep[pg. 220]{ep2e_1.1_2019}
    },
    sort={Durability}
}

\newglossaryentry{Ecto}{
    name={Ecto},
    plural={Ectos},
    description={
        “Ectos are mobile devices equivalent to mesh inserts, minus the internal medical sensors, used by bioconservatives or the implant-averse. They are flexible, dirt-resistant, and self-cleaning.\gPar{}
        They are often worn as bracelets or other adornments, and may be stretched out to increase screen size. They are coupled with wireless-enabled glasses, contact lenses, earpieces, and haptic gloves for experiencing augmented reality. Using an ecto applies a –10 modifier to mesh actions.” \citep[pg. 316]{ep2e_1.1_2019}
    },
    sort={Ecto}
}

\newglossaryentry{Egocasting}{
    name={Egocasting},
    plural={Egocastings},
    description={
        “Shuttlecraft using a variety of propulsion systems make regular trips between habitats, planetary surfaces, and moons. But for any trip longer than 1.5 million kilometers — the distance a fusion drive craft can cover in about a day — most people egocast.
        \gPar{}
        Egocasting is transhumanity’s most advanced personal transportation technology, though only your ego actually travels. Egocasting combines uploading ▶287 and quantum farcasting ▶336 to securely transfer your mind over interplanetary distances. This can be an active infomorph, an inactive backup, or even an upload transferred from your conscious mind.
        \gPar{}
        Though egocasting occurs at the speed of light, times vary drastically with distance. Egocasting within a cluster or planetary system is usually just a matter of minutes. Egocasting from the sun to the Kuiper Belt, however, takes between 40 and 70 hours, and so egocasting all of the way across the Solar System can take even longer. 
        \gPar{}
        Most egocasting is handled via legitimate farcasting services, often operated by the habitat’s government. If you are uploading from a morph, it may be sold, leased, or stored with a body bank service. Most people sell their morph, trading it in for a new sleeve at their destination. Storage/leasing is primarily reserved to the rich, people returning quickly, or people with a particular attachment to their morph (sometimes because it is their original).” \citep[pg. 300]{ep2e_1.1_2019}
    },
    sort={Egocasting}
}

\newglossaryentry{Ego Bridge}{
    name={Ego Bridge},
    plural={Ego Bridges},
    description={
        “Ego bridges are devices used for uploading from and downloading to biological brains  (Backups \& Uploading ▶286). The bridge’s cranial sensors unfold around your head when in use, imaging and scanning your brain. Needles in the neck rest deploy nanobots that either measure your mind and neural connections (uploading) or physically re-map them (downloading).” \citep[pg. 342]{ep2e_1.1_2019}
    },
    sort={Ego Bridge}
}

\newglossaryentry{Exploit}{
    name={Exploit},
    plural={Exploits},
    description={
        “A hacker library/tool for taking advantage of known software vulnerabilities. Required for hacking.” \citep[pg. 326]{ep2e_1.1_2019}
    },
    sort={Exploit}
}

\newglossaryentry{Explorenaut}{
    name={Explorenaut},
    plural={Explorenauts},
    description={
        “These small-sized bots travel on smart treads or with thrust-vector jets. They are loaded with sensors and favored for gatecrashing and similar exploration ops. A pair of manipulator arms are used for taking samples.” \citep[pg. 347]{ep2e_1.1_2019}
    },
    sort={Explorenaut}
}

\newglossaryentry{Fake Ego ID}{
    name={Fake Ego ID},
    plural={Fake Ego IDs},
    description={
        “This forged ID will pass in most inner system and Jovian Republic habitats, and sometimes others. It gives you a rep score in one network with that ID at 10.” \citep[pg. 315]{ep2e_1.1_2019}
    },
    sort={Fake Ego ID}
}

\newglossaryentry{Fork}{
    name={Fork},
    plural={Forks},
    description={...},
    sort={Fork}
}

\newglossaryentry{Forking}{
    name={Forking},
    plural={Forkings},
    description={
        “One of the repercussions of translating your mind into a digital format is that you can copy it just like any other file. Taking a copy of a transhuman ego and re-instancing it is called forking. To many, forking is an advantage, allowing you to multi-task and get more done. Need help on a project? Fork a few copies of yourself. Need to be in simultaneous meetings on Mars and Venus? Spin off a fork and egocast it to one while you handle the other — or send forks to both and go grab a drink.” \citep[pg. 292]{ep2e_1.1_2019}
    },
    sort={Forking}
}

\newglossaryentry{Ghostrider Module}{
    name={Ghostrider Module},
    plural={Ghostrider Modules},
    description={
        “This implant is a host for carrying another infomorph. This infomorph can be another muse, an ALI, a backed-up ego, or a fork. The module is linked to your mesh inserts, so the ghost-rider can mentally communicate with you, access the mesh, and connect to other parts of your PAN, depending on what access privileges you allow. You may install multiple modules.” \citep[pg. 320]{ep2e_1.1_2019}
    },
    sort={Ghostrider Module}
}

\newglossaryentry{Good Instincts}{
    name={Good Instincts},
    plural={Good Instincts},
    description={
        “Your gut feelings are on target. You get +5 per level to INT Checks.” \citep[pg. 74]{ep2e_1.1_2019}
    },
    sort={Good Instincts}
}

\newglossaryentry{Heavy Combat Armor}{
    name={Heavy Combat Armor},
    plural={Heavy Combat Armor},
    description={
        “These bulky and noticeable armor plates protect against heavy weaponry for serious combat operations. The shell’s mobility systems and power output are also modified to handle the extra mass.” \citep[pg. 214]{ep2e_1.1_2019}\gPar{}
        Ware Type: H, Energy: +16, Kinetic: +14, Complexity/GP: Maj/3.
    }
}

\newglossaryentry{Home Device}{
    name={Home Device},
    plural={Home Devices},
    description={...},
    sort={Home Device}
}

\newglossaryentry{Large Fabber}{
    name={Large Fabber},
    plural={Large Fabbers},
    description={
        “Large-sized fabbers are non-portable, unless mounted on a bot or vehicle, with a volume of 80 liters. They can manufacture most medium-sized items, or two or more of the same small-sized items at a time, or four or more identical very small objects at once.” \citep[pg. 343]{ep2e_1.1_2019}
    },
    sort={Minifac}
}

\newglossaryentry{Mesh}{
    name={Mesh},
    plural={Meshes},
    description={
        “Nearly every object is wireless and computerized. Each of these devices links to its neighbors to join the local network, sharing processing capabilities and functionality. This is the mesh; a hyper-connected, distributed, everyware network. The mesh offers free and unlimited digital processing power to its citizens. Digital agents crawl the mesh running errands for their masters. You wear or are implanted with mesh inserts that reveal a dazzling augmented reality overlay. Via entoptic displays, the world blossoms in an elaborate display that can only exist in a virtual world. Wild animals roam electronic storefronts, new fashions include moving dioramas.\gPar{}
        Every language is seamlessly translated, every moment is perfectly recorded, every question taps into the wealth of transhuman knowledge to produce an expert response. But the mesh's omnipresence has its downsides. Everything is hackable, and everything is watching; surveillance is everywhere and accessible to all.” \citep[p. 20]{ep2e_1.1_2019}
    },
    sort={Mesh}
}

\newglossaryentry{Mesh ID}{
    name={Mesh ID},
    plural={Mesh IDs},
    description={
        “Your mesh ID is a unique identifier code that distinguishes you from every other user and device on the mesh. Your cranial computer or ecto automatically associates this ID address with your accounts and screen names, enabling you to receive messages and other transmissions. This ID is automatically generated each time you come online and required for almost all online interactions. These interactions are universally logged, leaving a data trail that can be used to track your activity (Tracking ▶256). Fortunately, anonymizing services ▶315 are available for those who value their privacy. Many mesh IDs are publicly registered (and in some jurisdictions, this is legally mandated). Looking up a registered mesh ID is trivial. Registration data may include a name, physical address, and social media profile if so desired.” \citep[p. 246]{ep2e_1.1_2019}
    },
    sort={Mesh ID}
}

\newglossaryentry{MeshInsert}{
    name={Mesh Insert},
    plural={Mesh Inserts},
    description={
        “This network of implants is mandatory for people who want to use augmented reality and link wirelessly to the mesh. The various components include:\gPar{}
        \begin{itemize}
            \item \textbf{Cranial Computer:} This host serves as the hub for your personal area network and is home to your muse. It manages your augmented reality input and processes XP data, enabling you to share your sensorium with others in real-time. It is loaded with basic apps and provides all the functions you would expect from a mobile device: file storage, search engine, media player, mesh browser, address book, e-mail, messaging, and so forth.
            \item \textbf{Medical/ Diagnostic Sensors:} This array monitors your health, including heart rate, respiration, blood pressure, temperature, neural activity, ware status, and more. In synthmorphs, this system monitors system reports and error logs, heat, stress faults, and similar hardware statuses.
            \item \textbf{Radio Transceiver:} This connects your headware with other mesh devices within range (5 km urban areas/50 km open areas).
            \item You can access any of these functions simply by thinking.
        \end{itemize}” \citep[p. 20]{ep2e_1.1_2019}
    },
    sort={Mesh Insert}
}

\newglossaryentry{Minifac}{
    name={Minifac},
    plural={Minifacs},
    description={
        “Colloquially known as mini-factories or minifacturers, minifacs are huge, industrial-scale fabricators. These devices are used for much of the manufacturing throughout the Solar System, capable of making everything from synthmorphs to vehicles to habitats.” \citep[pg. 343]{ep2e_1.1_2019}
    },
    sort={Minifac}
}

\newglossaryentry{Muse}{
    name={Muse},
    plural={Muses},
    description={
        “Muses are a subtype of ALI designed to be personal aides and companions. Most transhumans grew up with a muse at their virtual side. Muses have more personality and psychological programming than standard ALIs and over time they build up an extensive database of their user’s preferences, likes and dislikes, and personality quirks so that they may more effectively be of service and anticipate needs. Most muses reside within their owner’s mesh inserts or ecto, where they can manage their owner’s PAN, communications, online searches, rep interactions, and other mesh activity.” \citep[pg. 250]{ep2e_1.1_2019}
    },
    sort={Muse}
}

\newglossaryentry{Nanofabricator}{
    name={Nanofabricator},
    plural={Nanofabricators},
    description={
        “Nanofabrication machines are universal assemblers. They can manufacture almost anything from the molecular level up, from a weapon to an ultralight plane to a hot and delicious dinner. You simply need raw materials (“feedstock”) and electronic plans (“blueprints”). Feedstock is readily available in most circumstances, distributed in blocks or by habitat utility feedlines, though it may be an issue in remote outposts or if your design requires exotic materials (Nanofabrication ▶314). Most fabricators are equipped with disassemblers for turning unwanted matter into feedstock. Disassemblers are programmed not to disassemble living tissue. Nanofabricators come equipped with a library of free common-use blueprints: basic foods, standard clothing, common tools, emergency gear, etc. Other blueprints must either be purchased online, self-programmed, or acquired through some other method. In capitalist economies, blueprints are licensed and protected by copyright laws. Pre-programmed restrictions will prevent you from using unlicensed blueprints or from manufacturing weapons, explosives, or other restricted items. Among the autonomists of the outer system, however, nanofabricators are commonly accessible, shared by everyone, and unrestricted, with entire libraries of open-source blueprints (many of them reverse-engineered from proprietary designs). Many different terms are used to describe nanofabricators: makers, compilers, forges, cornucopia machines, printers, and replicators, though fabbers remains the most commonplace. Fabbers are categorized by their size; each can make items of any smaller size. At the GM’s discretion, nanofabricators can create items of their size category or larger, but in pieces that must be assembled (possibly requiring an appropriate Hardware or other skill test to assemble).
        \begin{itemize}
            \item \textbf{Small Fabber:} These small and portable fabbers can produce objects up to a very small size with the appropriate blueprint. They have a maximum volume of 2 liters.
            \item \textbf{Medium Fabber:} These desktop fabbers can manufacture up to small-sized items. They may be able to print multiple of the same very small items at once. They have a maximum volume of 40 liters.
            \item \textbf{Large Fabber:} Large-sized fabbers are non-portable, unless mounted on a bot or vehicle, with a volume of 80 liters. They can manufacture most medium-sized items, or two or more of the same small-sized items at a time, or four or more identical very small objects at once.
            \item \textbf{Minifac:} Colloquially known as mini-factories or minifacturers, minifacs are huge, industrial-scale fabricators. These devices are used for much of the manufacturing throughout the Solar System, capable of making everything from synthmorphs to vehicles to habitats.
        \end{itemize}” \citep[p. 343]{ep2e_1.1_2019}
    }
}

\newglossaryentry{Obliviousness}{
    name={Obliviousness},
    plural={Obliviousness},
    description={
        “You are oblivious to events around you or anything other than what your attention is focused on. Suffer a –10 modifier to Perceive Tests against surprise and increase your distracted modifier to –30.” \citep[pg. 79]{ep2e_1.1_2019}
    },
    sort={Obliviousness}
}

\newglossaryentry{Operating System}{
    name={Operating System},
    plural={Operating Systems},
    description={
        “Your operating system (OS) is the software interface for your hardware device. It allows you to control hardware functions and manage other software resources, such as apps and services.” \citep[pg. 244]{ep2e_1.1_2019}
    },
    sort={Operating System}
}

\newglossaryentry{Oracles}{
    name={Oracles},
    plural={Oracles},
    description={
        “This neural macrosensing processor helps you pay attention to sensory input you are not focusing on, alerting you to important things you might otherwise overlook. Oracles provide a +10 bonus to Perceive and negate the distraction modifier for Perceive Tests.” \citep[pg. 319]{ep2e_1.1_2019}
    },
    sort={Oracles}
}

\newglossaryentry{Poor Coordination}{
    name={Poor Coordination},
    plural={Poor Coordination},
    description={
        “Either you or your morph are inherently clumsy. Suffer –5 per level to REF Checks.” \citep[pg. 79]{ep2e_1.1_2019}
    },
    sort={Poor Coordination}
}

\newglossaryentry{Privacy Mode}{
    name={Privacy Mode},
    plural={Privacy Modes},
    description={
        “If you’d prefer to keep a low profile, you can go into privacy mode, which provides a small degree of anonymity. Privacy mode hides your social profile — no one will be able to view your data, but they will still be able to anonymously ping and ding your rep scores. Privacy mode will also ask sensors and other devices to ignore and not track you, though depending on their configurations or AIs they may simply ignore this request. It applies a –30 modifier to mesh tracking attempts ▶256. Privacy mode is considered rude or gauche in some circumstances. An option to allow authenticated police/ security to override this mode is legally required in some authoritarian jurisdictions, but it may be easily toggled off.” \citep[pg. 241]{ep2e_1.1_2019}
    },
    sort={Privacy Mode}
}

\newglossaryentry{Private Server}{
    name={Private Server},
    plural={Private Servers},
    description={
        “Capable of running simulspace and 50 infomorphs.” \citep[pg. 315]{ep2e_1.1_2019}
    },
    sort={Private Server}
}

\newglossaryentry{Resleeving}{
    name={Resleeving},
    plural={Resleeving},
    description={
        “Digitized minds can also be downloaded. If your current body becomes damaged, obsolete, or inconvenient, you can sleeve into a new one. You can transfer consciousness with the right equipment and less than an hour’s time. Your new form could be an Olympic runner, a robotic shell, a digital presence, or even an orbiting shuttle. Members of high-risk professions, such as criminals or Firewall sentinels, can be killed multiple times on a mission only to be brought back and sent into the field again and again. Your body is called a sleeve or a morph, and the process of changing morphs is resleeving. Morphs can be biological with organic brains (biomorphs), biological with synthetic brains (pods), robotic (synthmorphs), or purely digital (infomorph).” \citep[pg. 20]{ep2e_1.1_2019}
        \gPar{}
        “Each time you resleeve, you must make two tests: an Integration Test and a Stress Test. The only pool that can be used on these tests is your ego’s Flex pool; pools from your morph do not apply.
        \gPar{}
        \textbf{Integration Test}\\
        The Integration Test determines how quickly you adjust to your new morph. Make a SOM Check, applying the modifier from your new morph’s Exotic Morphology trait, if any. If you succeed, you acclimate quickly. If you fail, you suffer –10 to all actions for 1 day, plus 1 day per superior failure.
        \gPar{}
        GMs should keep the PC’s original morph in mind when applying the Exotic Morphology modifiers. That trait is specifically assigned from a human-centric perspective (i.e., morphs that are less human have higher modifiers). However, this may not be fitting for an infolife or uplift PC. Adjust as appropriate.
        \gPar{}
        \textbf{Resleeving Stress Test}\\
        The Resleeving Stress Test incorporates all of the mentally challenging aspects of downloading into a new body into a single test: alienation, continuity, remembering death, and lack. Like other stress tests ▶229, make a WIL Check and apply modifiers as appropriate. The Stress Value is based on the most stressful aspect of resleeving. If this is standard alienation, continuity loss, and/or lack, the SV is 1d6. If you remember your death, or if you suffer a particularly long period of lack (over 3 months), the SV is 1d10/1. GMs should feel free to adjust these Stress Values as they see appropriate.” \citep[pg. 288]{ep2e_1.1_2019}
    },
    sort={Resleeving}
}

\newglossaryentry{Resources}{
    name={Resources},
    plural={Resources},
    description={
        “You have a measure of money, assets, and/or other wealth, as used in the inner system, hypercorp, Jovian, and Extropian polities. This provides bonus Morph Points and Gear Points equal to the trait's level when acquiring morphs ▶290 and gear ▶312. It also gives you a regular amount of disposable income to purchase gear during missions.\gPar{}
        At Level 1, you can spend up to 2 GP per week on Minor complexity items given the appropriate time frame.\gPar{}
        At Level 2, you can spend up to 3 GP per week on Minor or Moderate complexity items given the appropriate time frame.\gPar{}
        At Level 3, you can spend up to 5 GP per week on items of any complexity, given the appropriate time frame.\gPar{}
        Level 4 is the same as Level 3, except that you also have the capability to make even Rare and Restricted items available (at the gamemaster's discretion).\gPar{}
        In most cases, acquiring the gear is simply a matter of exploiting your Resources trait and waiting the proper amount of time, assuming the desired item is available (Acquiring Gear During Missions ▶312). At the GM’s discretion, however, using Resources may require an appropriate Persuade Test (to convince another party to part ways with the item) or perhaps a Research Test (to find a source).\gPar{}
        Levels 3 and 4 of this trait imply an amount of resources that deems you wealthy. To reflect this, you can use 2 of your weekly GP in conjunction with a Flex point for narrative control to say that you have a Moderate gear item immediately on hand in your home/ vehicle/ personal possessions. You must have access to your personal possessions and (as with all uses of Flex for narrative control) the item must be plausible.\gPar{}
        Resources can also apply as a modifier for certain tests. For example, if you attempt to bribe a triad goon or use your credit score to arrange a meeting with a potential business partner, apply a +10 modifier for each level of Resources you possess.\gPar{}
        While Resources is an abstract measurement, players and GMs should use it as a rough benchmark for a character’s personal assets and lifestyle. A character with Level 1 might have their own cubicle in a beehive hab or a small apartment in a Martian dome or O’Neill cylinder’s working class areas, and they get around by bike or public transit. A character with Level 2 Resources might have a private residence on a small station or a condo in a larger hab, as well as a minicar or cycle to get around. A character with Level 3 could have a large residential complex or multiple homes, plus one or more vehicles. A Level 4 character is rich and might own a small private hab and even their own shuttle.\gPar{}
        Your Resources trait may be affected by events in game. If your home is destroyed or you come across a secret cache of riches, the GM should adjust your trait level accordingly. You must pay the extra cost in Rez Points if your trait goes up, but you receive an RP credit if your wealth goes down.\gPar{}
        In desperate circumstances, you may also intentionally burn your Resources to refresh your weekly GP to get something you urgently need (or get it more quickly). This represents the expenditure of all or major portions of your assets with no hope of reclaiming them and no RP reimbursement. The GM should reduce your Resources trait level by an amount appropriate to the transaction.\gPar{}” \citep[p. 75]{ep2e_1.1_2019}
    },
    sort={Resources}
}

\newglossaryentry{Robomule}{
    name={Robomule},
    plural={Robomules},
    description={
        “These six-legged cargo drones are designed to carry large, non-portable gear, such as servers, healing vats, tool shops, etc. They also serve as general-purpose supply drones, with smart-material straps and webbing to hold items and an envirosealed pod to protect its load from the environment.” \citep[p. 349]{ep2e_1.1_2019}
    },
    sort={Robomule}
}

\newglossaryentry{Server}{
    name={Server},
    plural={Servers},
    description={
        “A server is a large-sized, non-portable computer, capable of running VR simulspaces and multiple infomorphs.” \citep[pg. 337]{ep2e_1.1_2019}
    },
    sort={Server}
}

\newglossaryentry{Simulspace}{
    name={Simulspace},
    plural={Simulspaces},
    description={
        “You have access to a virtual game environment, private meeting space, interactive media service, unreal vacation library, or other simulspace environment.” \citep[pg. 315]{ep2e_1.1_2019}
    },
    sort={Simulspace}
}

\newglossaryentry{Skillsoft}{
    name={Skillsoft},
    plural={Skillsofts},
    description={
        “These are skills encoded in software form. Used with a skillware system, they provide you with a rating of up to 40 in a single active skill or 80 in a Know skill (your aptitudes do not effect this rating; if you already possess the skill, use the highest value).” \citep[pg. 321]{ep2e_1.1_2019}\gPar{}
        Complexity/GP: Mod/2.
    },
    sort={Skillsoft}
}

\newglossaryentry{Skillware}{
    name={Skillware},
    plural={Skillwares},
    description={
        “Your brain is laced with a network of artificial neurons that can be formatted with information. This allows you to download \glsdisp{Skillsoft}{skillsofts} ▶below into your brain, gaining the use of those programmed skills until the \glsdisp{Skillsoft}{skillsoft} is erased or replaced. Skillware systems are only capable of handling 120 total skill points worth of skillsofts at a time. Switching out a \glsdisp{Skillsoft}{skillsoft} is a complex action.” \citep[pg. 320]{ep2e_1.1_2019}\gPar{}
        Ware Type: CHM, Complexity/GP: Maj/3.
    },
    sort={Skillware}
}

\newglossaryentry{Spoofer}{
    name={Spoofer},
    plural={Spoofers},
    description={
        “Use spoof apps to fake transmissions and mesh IDs (Spoofing ▶247).” \citep[pg. 326]{ep2e_1.1_2019}
    },
    sort={Spoofer}
}

\newglossaryentry{Structural Enhancement}{
    name={Structural Enhancement},
    plural={Structural Enhancements},
    description={
        “This modification bolsters the shell’s structural integrity, increasing its toughness and ability to take damage. Increase Wound Threshold by 2, Durability by 10, and Death Rating by 20.” \citep[pg. 323]{ep2e_1.1_2019}
    },
    sort={Structural Enhancement}
}

\newglossaryentry{Superior Numeracy}{
    name={Superior Numeracy},
    plural={Superior Numeracy},
    description={
        “You are quite good with numbers. Apply a +10 per level to Know and Technical Tests that directly involve math.” \citep[pg. 76]{ep2e_1.1_2019}
    },
    sort={Superior Numeracy}
}

\newglossaryentry{Superior Result}{
    name={Superior Result},
    plural={Superior Results},
    description={
        “Sometimes you will rock your test with flair, sometimes you will fumble it with indignity. On a roll of 33 or more that succeeds, you get a superior success. On a roll of 66 or more, you get two superior successes. On a roll of 66 or less that fails, you get a superior failure. On a roll of 33 or less, you get two superior failures.” \citep[pg. 31]{ep2e_1.1_2019}
    },
    sort={Superior Result}
}

\newglossaryentry{TacNet}{
    name={TacNet},
    plural={TacNets},
    description={
        “Tacnets allow a group and their muses/gear to share real-time tactical situational and sensory data over encrypted mesh channels. They are used by sports teams, security/military units, gamers, and anyone else that needs to coordinate actions. Tacnets provide the following functions: [...]” \citep[pg. 327]{ep2e_1.1_2019}
    },
    sort={TacNet}
}

\newglossaryentry{Teamwork}{
    name={Teamwork},
    plural={Teamwork},
    description={
        “Multiple characters may cooperate on a test, such as pushing open a door or repairing a robot together. These collaborators must reasonably be able to communicate and work together in an efficient fashion. Only one character rolls for the test (usually the one with the highest skill). Each additional contributing character provides a +10 modifier, up to a maximum of +30. Only the character making the test can use pools ▶34 to affect the test. For tests involving Know, Technical, or Vehicle skills ▶48, collaborating characters must possess the tested skill at 40 or more to provide a teamwork bonus.” \citep[pg. 31]{ep2e_1.1_2019}
    },
    sort={Teamwork}
}

\newglossaryentry{Topology}{
    name={Topology},
    plural={Topologies},
    description={
        “Though the mesh as a whole consists of innumerable devices all connected to the other devices around them, the actual layout is more complex. Many sub-networks exist within the mesh network itself: slaved devices, VPNs, PANs, and tiered systems.” \citep[pg. 241]{ep2e_1.1_2019}
    },
    sort={Topology}
}

\newglossaryentry{Tracker}{
    name={Tracker},
    plural={Tracker},
    description={
        “This app traces people’s connections online to their origin (Tracking ▶256).” \citep[pg. 326]{ep2e_1.1_2019}
    },
    sort={Tracker}
}

\newglossaryentry{VPN}{
    name={VPN},
    plural={VPNs},
    description={
        “This app enables you to communicate over a virtual private network (VPNs ▶241). VPNs provide a –30 modifier to sniffing attacks (Sniffing ▶245).” \citep[pg. 326]{ep2e_1.1_2019}
    },
    sort={VPN}
}

\newglossaryentry{Ware}{
    name={Ware},
    plural={Wares},
    description={
        “Ware is a catch-all category for augmentations of different kinds:
        \begin{itemize}
            \item \textbf{Bioware (B)} includes genetic modifications, nanosurgical tissue alterations, and implantation of bio-engineered organs. It is only available for biomorphs (including pods and uplifts). Because of its organic nature, bioware is hard to detect in scans; genetic testing or other bio-sampling is required. Use Medicine: Biotech to diagnose, implant, or repair.
            \item \textbf{Cyberware (C)} is synthetic devices either implanted within a biological body or “grown” within using nanobots. It is only available for biomorphs (including pods and uplifts). Cyberware is detectable with x-ray and radar scans. It can also be hacked like other electronics. Use Medicine: Biotech to implant, Hardware: Electronics to repair the ware itself.
            \item \textbf{Hardware (H)} includes add-ons to synthetic shells. It is only available for synthmorphs, bots, and vehicles. It can be hacked. Use Hardware: Robotics or an appropriate field for vehicles to install, diagnose, and repair. Many non-ware gear items can be mounted on or incorporated into a shell’s frame as hardware (GM discretion).
            \item \textbf{Meshware (M)} refers to plug-in apps that enhance infomorphs. These may also be installed with cyberbrains. They are vulnerable to mesh combat. Use Program to install, diagnose, or repair.
            \item \textbf{Nanoware (N)} refers to internal nanobot systems active within a body or shell. Nanoware includes an implanted hive that maintains and refreshes the nanobots. It is available for morphs of all types. Nanoware hives are detectable like cyberware, but the bots are only detectable with nanodetectors or detailed biological sampling. Nanoware can be hacked like other electronics. Use Medicine: Biotech or Hardware: Robotics to implant/repair.
            \gPar{}
            Unless otherwise noted, each ware item can only be installed in the same morph once, no matter if it is available in different forms.
        \end{itemize}” \citep[pg. 311]{ep2e_1.1_2019}
    },
    sort={Ware}
}

\newglossaryentry{Ware Type}{
    name={Ware Type},
    plural={Ware Types},
    description={see \gls{Ware}},
    sort={Ware Type}
}
