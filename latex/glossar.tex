\newglossaryentry{Active Defender}{
    name={Active Defender},
    description={...},
    sort={Active Defender}
}

\newglossaryentry{AGI}{
    name={AGI},
    plural={AGIs},
    description={
        Artificial General Intelligence. An artificial intelligence that is in the human sense generally intelligent, meaning it can handle tasks it was not trained for in a common sense intelligent manner.
    },
    sort={AGI}
}

\newglossaryentry{Blueprint}{
    name={Blueprint},
    plural={Blueprints},
    description={
        “In order to print something on a nanofabricator, you need a blueprint. Most have a small inventory of essential items. Others may have blueprints in storage, but locked to specific users (these can be hacked per standard Infosec intrusion tests).\gPar{}
        Nanofab blueprints come in three forms: single use, multi use, and open source,
        \begin{itemize}
            \item \textbf{Single-Use Blueprints:} Most single-use blueprints are accessed via a digital distribution platforms which use digital rights management (DRM) to protect copyright. This means you can access the file to print it once, but it cannot be copied or printed again. If you need a file you can print without mesh access, some single-use blueprints are available as downloadable digital files that block efforts to copy them and erase themselves after use. The inner-system hypercorps and capitalist economies use these methods to control scarcity and keep you dependent. Single-use blueprints are acquired with GP, rep, or Resources trait, using the item’s complexity.
            \item \textbf{Limited-Use Blueprints:} A few hypercorps and transitional economies allow limited licenses. These work the same as single-use blueprints, except that if you recycle the item by disassembling it in a licensed fabber, you receive a credit which you can then use to print the item again. This is ideal for those who egocast often but wish to bring their gear with them.
            \item \textbf{Multi-Use Blueprints:} A multi-use blueprint is more akin to purchasing a personal license. You can print the item repeatedly and copy the blueprint, but it is keyed to your ID. These are also acquired via GP, rep, or Resources trait. For a non-erasing, re-usable, multi-use blueprint, increase the complexity by one step (Minor becomes Moderate, Moderate becomes Major). Multi-use blueprints for Major complexity items are rare and largely unavailable.
            \item \textbf{Open-Source Blueprints:} In autonomist regions and many other areas, some blueprints are freely available from open-source libraries. These blueprints are re-usable, can be copied, and do not require your authenticated ID to use. A Research Test (using the Rep Modifiers on the Rep Tests table ▶308) or (@-rep, i-rep, r-rep, or x-rep) favor equal to the item’s complexity plus one step will get you an open-source blueprint, even for things like weapons and drugs. The drawback is that open-source prints are sometimes less reliable, more experimental, or carry hidden malware payloads. Every open-source blueprint acquired via Research (not rep) has a 20\% chance of being unreliable: a –10 to –30 modifier, unexpected bugs, dangerous malware, etc. These blueprints are also illegal in certain inner-system polities (many are cracked proprietary designs) and may not work on inner-system fabbers.
        \end{itemize}” \citep[pg. 314]{ep2e_1.1_2019}
    }
}

\newglossaryentry{Cortical Stack}{
    name={Cortical Stack},
    plural={Cortical Stacks},
    description={
        “This diamond memory module is implanted at the base of the skull, where the brain stem and spinal cord connect (or in similar locations on synthmorphs). It is linked to a neural lace network of nanobots saturating the brain that monitor synaptic connections, brain architecture, and chemical levels. It effectively takes a snapshot of your brain every second, making a digital backup of your ego, right up until the moment you die. It also backs up your muse. If you are killed, this stack can be used to retrieve a backup of your ego ▶287 so that you can be re-instantiated. Cortical stacks do not have external or wireless access (for security purposes), they must be surgically removed.” \citep[pg. 316]{ep2e_1.1_2019}
    },
    sort={Cortical Stack}
}

\newglossaryentry{Crypto}{
    name={Crypto},
    plural={Cryptos},
    description={
        “This app generates key pairs, encrypts messages using
    public keys, and decrypts with secret keys (Encryption ▶247).” \citep[pg. 326]{ep2e_1.1_2019}
    },
    sort={Crypto}
}

\newglossaryentry{Cyberbrain}{
    name={Cyberbrain},
    plural={Cyberbrains},
    description={
        “Cyberbrains are specialized hardware for running virtual mind-states, allowing an ego or ALI to control a pod or synthmorph. Modeled on biological brains, cyberbrains have a holistic architecture and serve as the command node and central processing point for sensory input and decision-making. As hosts, only one infomorph can inhabit a cyberbrain at a time. Any ego within a cyberbrain can move or copy itself as an infomorph to another device. By default, your ego will manifest in a digimorph ▶67, unless you have another type of infomorph available. Cyberbrains are vulnerable to hacking (Mindware Hacking ▶266). All cyberbrains are equipped with access jacks, mnemonics, and a puppet sock for remote control, at no additional cost.” \citep[pg. 316]{ep2e_1.1_2019}
    },
    sort={Cyberbrain}
}

\newglossaryentry{Drone Rig}{
    name={Drone Rig},
    plural={Drone Rigs},
    description={
        “This simsense augmentation gives you better control when jamming drones (Remote Operations ▶346). You ignore the –10 modifier for jamming.” \citep[pg. 320]{ep2e_1.1_2019}
    },
    sort={Dron Rig}
}

\newglossaryentry{Ego Bridge}{
    name={Ego Bridge},
    plural={Ego Bridges},
    description={
        “Ego bridges are devices used for uploading from and downloading to biological brains  (Backups \& Uploading ▶286). The bridge’s cranial sensors unfold around your head when in use, imaging and scanning your brain. Needles in the neck rest deploy nanobots that either measure your mind and neural connections (uploading) or physically re-map them (downloading).” \citep[pg. 342]{ep2e_1.1_2019}
    },
    sort={Ego Bridge}
}

\newglossaryentry{Exploit}{
    name={Exploit},
    plural={Exploits},
    description={
        “A hacker library/tool for taking advantage of known software vulnerabilities. Required for hacking.” \citep[pg. 326]{ep2e_1.1_2019}
    },
    sort={Exploit}
}

\newglossaryentry{Fake Ego ID}{
    name={Fake Ego ID},
    plural={Fake Ego IDs},
    description={
        “This forged ID will pass in most inner system and Jovian Republic habitats, and sometimes others. It gives you a rep score in one network with that ID at 10.” \citep[pg. 315]{ep2e_1.1_2019}
    },
    sort={Fake Ego ID}
}

\newglossaryentry{Fork}{
    name={Fork},
    plural={Forks},
    description={...},
    sort={Fork}
}

\newglossaryentry{Home Device}{
    name={Home Device},
    plural={Home Devices},
    description={...},
    sort={Home Device}
}

\newglossaryentry{Mesh}{
    name={Mesh},
    plural={Meshes},
    description={
        “Nearly every object is wireless and computerized. Each of these devices links to its neighbors to join the local network, sharing processing capabilities and functionality. This is the mesh; a hyper-connected, distributed, everyware network. The mesh offers free and unlimited digital processing power to its citizens. Digital agents crawl the mesh running errands for their masters. You wear or are implanted with mesh inserts that reveal a dazzling augmented reality overlay. Via entoptic displays, the world blossoms in an elaborate display that can only exist in a virtual world. Wild animals roam electronic storefronts, new fashions include moving dioramas.\gPar{}
        Every language is seamlessly translated, every moment is perfectly recorded, every question taps into the wealth of transhuman knowledge to produce an expert response. But the mesh's omnipresence has its downsides. Everything is hackable, and everything is watching; surveillance is everywhere and accessible to all.” \citep[p. 20]{ep2e_1.1_2019}
    },
    sort={Mesh}
}

\newglossaryentry{Nanofabricator}{
    name={Nanofabricator},
    plural={Nanofabricators},
    description={
        “Nanofabrication machines are universal assemblers. They can manufacture almost anything from the molecular level up, from a weapon to an ultralight plane to a hot and delicious dinner. You simply need raw materials (“feedstock”) and electronic plans (“blueprints”). Feedstock is readily available in most circumstances, distributed in blocks or by habitat utility feedlines, though it may be an issue in remote outposts or if your design requires exotic materials (Nanofabrication ▶314). Most fabricators are equipped with disassemblers for turning unwanted matter into feedstock. Disassemblers are programmed not to disassemble living tissue. Nanofabricators come equipped with a library of free common-use blueprints: basic foods, standard clothing, common tools, emergency gear, etc. Other blueprints must either be purchased online, self-programmed, or acquired through some other method. In capitalist economies, blueprints are licensed and protected by copyright laws. Pre-programmed restrictions will prevent you from using unlicensed blueprints or from manufacturing weapons, explosives, or other restricted items. Among the autonomists of the outer system, however, nanofabricators are commonly accessible, shared by everyone, and unrestricted, with entire libraries of open-source blueprints (many of them reverse-engineered from proprietary designs). Many different terms are used to describe nanofabricators: makers, compilers, forges, cornucopia machines, printers, and replicators, though fabbers remains the most commonplace. Fabbers are categorized by their size; each can make items of any smaller size. At the GM’s discretion, nanofabricators can create items of their size category or larger, but in pieces that must be assembled (possibly requiring an appropriate Hardware or other skill test to assemble).
        \begin{itemize}
            \item \textbf{Small Fabber:} These small and portable fabbers can produce objects up to a very small size with the appropriate blueprint. They have a maximum volume of 2 liters.
            \item \textbf{Medium Fabber:} These desktop fabbers can manufacture up to small-sized items. They may be able to print multiple of the same very small items at once. They have a maximum volume of 40 liters.
            \item \textbf{Large Fabber:} Large-sized fabbers are non-portable, unless mounted on a bot or vehicle, with a volume of 80 liters. They can manufacture most medium-sized items, or two or more of the same small-sized items at a time, or four or more identical very small objects at once.
            \item \textbf{Minifac:} Colloquially known as mini-factories or minifacturers, minifacs are huge, industrial-scale fabricators. These devices are used for much of the manufacturing throughout the Solar System, capable of making everything from synthmorphs to vehicles to habitats.
        \end{itemize}” \citep[p. 343]{ep2e_1.1_2019}
    }
}

\newglossaryentry{Oracles}{
    name={Oracles},
    plural={Oracles},
    description={
        “This neural macrosensing processor helps you pay attention to sensory input you are not focusing on, alerting you to important things you might otherwise overlook. Oracles provide a +10 bonus to Perceive and negate the distraction modifier for Perceive Tests.” \citep[pg. 319]{ep2e_1.1_2019}
    },
    sort={Oracles}
}

\newglossaryentry{Private Server}{
    name={Private Server},
    plural={Private Servers},
    description={
        “Capable of running simulspace and 50 infomorphs.” \citep[pg. 315]{ep2e_1.1_2019}
    },
    sort={Private Server}
}

\newglossaryentry{Simulspace}{
    name={Simulspace},
    plural={Simulspaces},
    description={
        “You have access to a virtual game environment, private meeting space, interactive media service, unreal vacation library, or other simulspace environment.” \citep[pg. 315]{ep2e_1.1_2019}
    },
    sort={Simulspace}
}

\newglossaryentry{Skillsoft}{
    name={Skillsoft},
    plural={Skillsofts},
    description={
        “These are skills encoded in software form. Used with a skillware system, they provide you with a rating of up to 40 in a single active skill or 80 in a Know skill (your aptitudes do not effect this rating; if you already possess the skill, use the highest value).” \citep[pg. 321]{ep2e_1.1_2019}\gPar{}
        Complexity/GP: Mod/2.
    },
    sort={Skillsoft}
}

\newglossaryentry{Skillware}{
    name={Skillware},
    plural={Skillwares},
    description={
        “Your brain is laced with a network of artificial neurons that can be formatted with information. This allows you to download \glsdisp{Skillsoft}{skillsofts} ▶below into your brain, gaining the use of those programmed skills until the \glsdisp{Skillsoft}{skillsoft} is erased or replaced. Skillware systems are only capable of handling 120 total skill points worth of skillsofts at a time. Switching out a \glsdisp{Skillsoft}{skillsoft} is a complex action.” \citep[pg. 320]{ep2e_1.1_2019}\gPar{}
        Ware Type: CHM, Complexity/GP: Maj/3.
    },
    sort={Skillware}
}

\newglossaryentry{Spoofer}{
    name={Spoofer},
    plural={Spoofers},
    description={
        “Use spoof apps to fake transmissions and mesh IDs (Spoofing ▶247).” \citep[pg. 326]{ep2e_1.1_2019}
    },
    sort={Spoofer}
}

\newglossaryentry{TacNet}{
    name={TacNet},
    plural={TacNets},
    description={
        “Tacnets allow a group and their muses/gear to share real-time tactical situational and sensory data over encrypted mesh channels. They are used by sports teams, security/military units, gamers, and anyone else that needs to coordinate actions. Tacnets provide the following functions: [...]” \citep[pg. 327]{ep2e_1.1_2019}
    },
    sort={TacNet}
}

\newglossaryentry{Tracker}{
    name={Tracker},
    plural={Tracker},
    description={
        “This app traces people’s connections online to their origin (Tracking ▶256).” \citep[pg. 326]{ep2e_1.1_2019}
    },
    sort={Tracker}
}

\newglossaryentry{VPN}{
    name={VPN},
    plural={VPNs},
    description={
        “This app enables you to communicate over a virtual private network (VPNs ▶241). VPNs provide a –30 modifier to sniffing attacks (Sniffing ▶245).” \citep[pg. 326]{ep2e_1.1_2019}
    },
    sort={VPN}
}
